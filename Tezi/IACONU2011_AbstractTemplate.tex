\documentclass[12pt]{article}
\usepackage{amsmath,amssymb, amsfonts, amsthm,amscd}
\usepackage[T2A]{fontenc}
\usepackage[cp1251]{inputenc}
\usepackage[russian,ukrainian,english]{babel}
\usepackage{graphicx}

\pagestyle{empty}

\textheight=235mm \textwidth=155mm \voffset=-8mm \hoffset=-14mm
\sloppy

\newtheorem{proposition}{Proposition}
\newtheorem{lemma}{Lemma}
\newtheorem{theorem}{Theorem}
\newtheorem{corollary}{Corollary}
\newtheorem*{proposition*}{Proposition}
\newtheorem*{lemma*}{Lemma}
\newtheorem*{theorem*}{Theorem}
\newtheorem*{corollary*}{Corollary}

\newtheorem{definition}{Definition}
\newtheorem{remark}{Remark}
\newtheorem{example}{Example}
\newtheorem*{definition*}{Definition}
\newtheorem*{remark*}{Remark}
\newtheorem*{example*}{Example}

\newcommand{\address}[2]
{\vspace{-3mm}\medskip
\begin{minipage}[t]{10.5cm}
\small{#1 \tt{#2}}
\end{minipage}}

\renewcommand\refname{\large \textbf{References}}

\begin{document}


\begin{center}
{\bf\large On definitions and basic properties of projective
modules}

\vskip 0.3cm \centerline{\normalsize R.~Hendersen, P.M.~Jefferson}
\vskip 0.3cm

{\it Ohio State University}

e-mail:qwertyuiop2013@gmail.com

\end{center}

This paper is an example of abstracts. Definitions, theorems,
lemmas, statements, corollaries, remarks, examples - all this
constructions can be implemented and used in your abstracts. The
following couple of theorems are the examples of mentioned
constructions.

\begin{theorem}\label{th.1}\cite{item1,item2,item3}
Let $R$ be any commutative ring with identity. Then following
statements about $R$-module $P$ are equivalent:

1) $P$ is projective module.

2) There are such free $R$-module $F$ and its submodule $Q$ that
$F=P\oplus Q$.

3) Every short exact sequence of modules of the form
$0\longrightarrow A\longrightarrow B\longrightarrow P
\longrightarrow 0$ is a split exact sequence.
\end{theorem}

\begin{corollary}\label{cor.1}
Every free module is projective.
\end{corollary}

\begin{example}\label{ex.1}
Over a direct product of rings $R\times S$ where $R$ and $S$ are
nonzero rings, both $R \times 0$ and $0 \times S$ are non-free
projective modules.
\end{example}

\begin{lemma}\label{lm.1}\cite{item1}
Every direct summand of projective $R$-module $P$ is also
projective.
\end{lemma}

For arbitrary ring $R$ submodules of right regular module $R_R$ are
its right ideals. Since $R_R$ is free as module, then it is
projective, by corollary~\ref{cor.1}. As a consequence we have:

\begin{corollary}\label{cor.2}
If a right ideal $I$ of ring $R$ is a direct summand of $R$ then its
is projective as right $R$-module.
\end{corollary}

Further investigations can be presented in the same manner.

\vspace{-5mm}\small
\begin{thebibliography}{10}
\bibitem{item1} Nicolas Bourbaki, Commutative algebra, Ch. II, �5.
\bibitem{item2} Paul M. Cohn (2003). Further algebra and applications. Springer. ISBN 1-85233-667-6.
\bibitem{item3} Serge Lang (1993). Algebra (3rd ed.). Addison�Wesley. ISBN 0-201-55540-9.
\end{thebibliography}


\end{document}
