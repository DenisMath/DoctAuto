\documentclass[12pt]{article}
\usepackage{amsmath,amssymb, amsfonts, amsthm,amscd}
\usepackage[T2A]{fontenc}
\usepackage[cp1251]{inputenc}
\usepackage[russian,ukrainian,english]{babel}
\usepackage{graphicx}

\pagestyle{empty}

\textheight=235mm \textwidth=155mm \voffset=-8mm \hoffset=-14mm
\sloppy

\newtheorem{proposition}{Proposition}
\newtheorem{lemma}{Lemma}
\newtheorem{theorem}{Theorem}
\newtheorem{corollary}{Corollary}
\newtheorem*{proposition*}{Proposition}
\newtheorem*{lemma*}{Lemma}
\newtheorem*{theorem*}{Theorem}
\newtheorem*{corollary*}{Corollary}

\newtheorem{definition}{Definition}
\newtheorem{remark}{Remark}
\newtheorem{example}{Example}
\newtheorem*{definition*}{Definition}
\newtheorem*{remark*}{Remark}
\newtheorem*{example*}{Example}

\newcommand{\address}[2]
{\vspace{-3mm}\medskip
\begin{minipage}[t]{10.5cm}
\small{#1 \tt{#2}}
\end{minipage}}

\renewcommand\refname{\large \textbf{References}}

\begin{document}


\begin{center}
{\bf\large Recursive criterion of conjugation of finite-state binary tree's automorphisms}

\vskip 0.3cm \centerline{\normalsize D.~Morozov}
\vskip 0.3cm

{\it National University of ``Kyiv-Mohyla Academy''}

e-mail: denis.morozov178@gmail.com

\end{center}
 In this paper conjugation problem in the group of finite-state automorphisms of rooted binary tree investigated.

\begin{definition}
Define $FAutT_2$ as group of finite-state automorphisms of rooted binary tree.
\end{definition}

\begin{definition}\label{marked_type_tree_def}
Denote {\bf the marked tree } for automorphism $f \in AutZ_2$  like this.

\begin{itemize}
\item The root of the tree  note by automorphism $f$.
\item If the vertex of the n-th level of the marked-type tree marked automorphism $a = (b,c)\circ\sigma$, then 
 only one edge  connects the n+1- th level with this vertex. Other vertex of this edge marked with automorphism $\pi_L(a)\circ \pi_R(a)$.
\item If the vertex of the n-th level of the marked-type tree marked automorphism $a = (b,c)$, then 
 two edges  connect the n+1- th level with this vertex. Other vertex of one edge marked with automorphism $\pi_L(a)$ and another edge marked with $\pi_R(a)$.

\end{itemize}
Automorphism that marked the vertex $ t \ in D_f $ of  marked-type tree denote as $ D_f (t) $.
The set of vertices of n-th level of tree $D$ denote as $L_n(D)$.
\end{definition}

\begin{lemma}\label{unique_solution_upper_vertex}
Let $$a = (a_1,a_2)\circ\sigma, b = (b_1,b_2)\circ\sigma$$ $$a' = a_1\circ a_2 ,b' = b_1\circ b_2 $$ 
If  $a'$ and $b'$ conjugated in $FAutT_2$
then $a$ and $b$ conjugated in $FAutT_2$.
\end{lemma}

\begin{theorem}\label{marked_trees_conj}
Automorphisms $a$ and $b$ conjugated in $FAutT_2$ if, and only if
$$\forall t\in L_n(D_a), \exists x\in FAutT_2,\ D_a(t)^x= D_b(t*\alpha) $$
\end{theorem}

\begin{corollary}\label{marked_trees_conj_level}
Automorphisms $a$ and $b$ conjugated in $FAutT_2$ if, and only if
$$\exists n\in\mathbb{N}, \forall t\in L_n(D_a), \exists x\in FAutT_2\ D_a(t)^x= D_b(t*\alpha) $$
\end{corollary}


This techniques applied for finite-state conjugation problem solving of differentiable  finite-state izometries of the ring of integer 2-adic numbers.

\vspace{-5mm}\small
\begin{thebibliography}{10}
\bibitem{D46} Denis Morozov. {Differentiable finite-state izometries and izometric
polynomials of the ring of integer 2-adic numbers}. 8th International Algebraic Conference, 
July 5-12 (2011), Lugansk, Ukraine.
\end{thebibliography}


\end{document}
